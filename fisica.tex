\documentclass{article}
\usepackage[utf8]{inputenc}
\usepackage[spanish]{babel}
\usepackage{amsmath}
\usepackage{amsfonts}
\usepackage{amsthm}
\usepackage{mathtools}
\usepackage{amssymb}
\newcommand{\norm}[1]%
{\ensuremath\left\vert{\!\!}\left\vert#1\right\vert{\!\!}\right\vert}

\title{Física}
\author{matesdades + victor}
\date{\today}

\begin{document}

\maketitle

\section{Cálculo vectorial}
Los vectores en física son $n$-tuplas de 1, 2 o 3 coordenadas. En dos 
dimensiones, frente a una transformación de coordenadas de rotación, los 
vectores se transforman por la matriz de rotación
\[
	\begin{pmatrix}
		\cos \theta & \sin \theta \\
		-\sin \theta & \cos \theta \\
	\end{pmatrix}\,,
\]
donde $\theta$ es el ángulo de rotación.

El producto escalar de dos vectores $v,w$ es
\[
	\vec{v}\cdot \vec{w} = \norm{\vec{v}}\norm{\vec{w}} \cos \theta = \sum_{i} 
	v_iw_i\,,
\]
donde $\theta$ es el ángulo más pequeño que forman los dos vectores.

El producto vectorial de dos vectores tridimensionales es
\[
	\vec{v}\times\vec{w} =
	\begin{vmatrix}
		\hat{i} & \hat{j} & \hat{k}\\
		v_1 & v_2 & v_3\\
		w_1 & w_2 & w_3
	\end{vmatrix}
	 = \left(\sum_{j,k}^{} \varepsilon_{ijk}v_jw_k\right)_{i\in(1,2,3)}\,,
\]
donde $\varepsilon_{ijk}$ es el símbolo de Levi-Civita.

\section{Cinemática}

\subsection{Movimiento curvilíneo}
\subsection{Coordenadas polares}
\subsection{Coordenadas intrínsecas}

\section{Dinámica}

\section{Energía}

\end{document}
